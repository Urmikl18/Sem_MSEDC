\documentclass[10pt, conference, compsocconf]{IEEEtran}

\begin{document}

\title{Adaptive Indexing: Fundamental and Hybrid Approaches}
\author{Pavlo Shevchenko \\ Otto-von-Guericke-University, Magdeburg \\ pavlo.shevchenko@st.ovgu.de}

\maketitle

\begin{abstract}
What did I do in a nutshell?
\end{abstract}

\section{Introduction}
Motivation. Main Idea. Goal. Structure of the paper.

\section{Background}
Idea of adaptive indexing. First approaches.

\section{Database Cracking}
\subsection{Introduction}
Short description of method 
\subsection{Strength of Database Cracking}
Small overhead + further advantages
\subsection{Weakness of Database Cracking}
Slow convergence + further disadvantages

\section{Adaptive Merging}
\subsection{Introduction}
Short description of method
\subsection{Strength of Adaptive Merging}
Fast convergence + further advantages
\subsection{Weakness of Adaptive Merging}
Big overhead + further disadvantages

\section{Hybrid Approaches}
\subsection{Strategies for designing hybrid approach}
Idea of perfect hybrid. Combination options. Further subsection need to be added.

\section{Evaluation}
Point out complementary nature of cracking and merging. Compare to other hybrid approaches. Speculate on future and usage of the methods

\section{Related Work}
Some research on related work has to be done.

\section{Conclusions}
What did I find out?

\section{Discussion}
What does it mean?

\section*{Acknowledgement}
I thank M.Sc. Gabriel Campero Durand of Otto-von-Guericke-University, Magdeburg for providing insight and expertise to start this research and for his guidance through the whole process of research, writing and evaluation of this scientific work. I would also like to show my gratitude to the DBSE Research Group of Otto-von-Guericke-University, Magdeburg for making this work possible and organising "Seminar on Modern Software Engineering and Database Concepts", during which this research took place.

\begin{thebibliography}{6}

\bibitem{cracking}
Idreos, Stratos, Martin L. Kersten, and Stefan Manegold. "Database Cracking." CIDR. Vol. 7. 2007.
\bibitem{survey_cracking}
Schuhknecht, Felix Martin. "Closing the circle of algorithmic and system-centric database optimization: a comprehensive survey on adaptive indexing, data partitioning, and the rewiring of virtual memory." (2016).
\bibitem{merging}
Graefe, Goetz, and Harumi Kuno. "Self-selecting, self-tuning, incrementally optimized indexes." Proceedings of the 13th International Conference on Extending Database Technology. ACM, 2010.
\bibitem{hybrid_approaches}
Idreos, Stratos, et al. "Merging what's cracked, cracking what's merged: adaptive indexing in main-memory column-stores." Proceedings of the VLDB Endowment 4.9 (2011): 586-597.
\bibitem{eval1}
Pirk, Holger, et al. "Database cracking: fancy scan, not poor man's sort!." Proceedings of the Tenth International Workshop on Data Management on New Hardware. ACM, 2014.
\bibitem{eval2}
Schuhknecht, Felix Martin, Alekh Jindal, and Jens Dittrich. "The uncracked pieces in database cracking." Proceedings of the VLDB Endowment 7.2 (2013): 97-108.

\end{thebibliography}

\end{document}